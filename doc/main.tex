% Created 2022-03-07 Mon 18:36
% Intended LaTeX compiler: pdflatex
\documentclass[platex,12pt,a4paper]{jsarticle}
\usepackage[utf8]{inputenc}
\usepackage[T1]{fontenc}
\usepackage{graphicx}
\usepackage{longtable}
\usepackage{wrapfig}
\usepackage{rotating}
\usepackage[normalem]{ulem}
\usepackage{amsmath}
\usepackage{amssymb}
\usepackage{capt-of}
\usepackage{hyperref}
\usepackage[margin=15mm]{geometry}
\usepackage{rmss_math}
\usepackage{titlesec}
\titleformat{\section}{\normalfont\Large\bfseries}{第\thesection{}講 }{0em}{}
\author{長江 剛志}
\date{\today}
\title{経済数学\\\medskip
\large 数理最適化と均衡分析の立場から}
\begin{document}

\maketitle
\tableofcontents

\section{イントロダクション}
\label{sec:org303a2b8}
\subsection{ラーメン屋と最適化・均衡状態}
\label{sec:org0c472b2}
\begin{itemize}
\item 2つのラーメン屋A軒とB亭があるとしよう.A軒はカウンターだけで狭いが,料金が安くボリュームたっぷりで学生に人気がある.
B亭は広くてテーブル席も多いが,料金の割に丼は小振りでファミリーに人気がある.
\item あなたが昼食にラーメンを食べたいと思い,A軒とB亭のどちらにしようかと考え,自分にとって望ましい方を選択することは, \textbf{\textbf{最適化}} と呼ばれる.
\begin{itemize}
\item たとえば,A軒とB亭とで,i)ラーメンが出てくるまでにかかる待ち時間; ii)料金; iii)味や量の好み,という3つの要素が次のようになっていたとする:
\begin{center}
\begin{tabular}{lll}
\hline
 & A軒 & B亭\\
\hline
待ち時間 & 30分 & 10分\\
料金 & 500円 & 700円\\
味, 量 & こってり, 350g(大盛り) & あっさり, 200g(普通)\\
\hline
\end{tabular}
\end{center}
\item これを最適化として取り扱うため,A軒とB亭のそれぞれでラーメンを食べることの「幸せの度合い」を数値化することにしよう.
数値化された幸せの度合いは, \textbf{\textbf{効用}} (utility)と呼ばれる.本来,効用には単位は必要ない
\footnote{
一般に,経済学では,効用はその大小(順番)にのみ意味があるという考え方をする(序数的効用).
つまり,AとBの選択肢があったとき「Aの効用がBの効用より大きい」ことにだけ意味があるのであって,そのことが表現できるならば,
Aの効用が \(100\) でBの効用が \(90\) であろうと,Aの効用が \(-1000\) でBの効用が \(-1001\)だろうと構わない,
という立場である.
これに対して,ここで示したように,効用の数値そのものに意味があるという考え方は,基数的効用と呼ばれる.
}
のだが,簡単のため,単位を「円」で考えることにする.
\item まず,各ラーメン屋の「味と量」に対する \textbf{\textbf{価値}} を考えよう.ここでの価値\(V\)は,
「そのラーメンになら\(V\)円までなら支払っても良いと思う(支払い意思額)」あるいは
「そのラーメンが\(V\)円以上だったら食べるのを諦める(留保価格)」とする.
この値段は,実際のメニュー表に載っている値段とは無関係に,あなたの主観で決まる.
ものすごくお腹が空いていたらA軒のラーメンに \(1,000\)円の価値を感じるかもしれないし,
胃がもたれていたらB亭のラーメンに \(750\)円の価値を感じるかもしれない.
注意したいのは,あなたが実際にお金を払ってでもラーメンを食べたい,というのは,
そのラーメンに対する価値が,そのラーメンの費用(後述)よりも高いことを意味することだ.
\item 次に,それぞれのラーメン屋の待ち時間も「円」に換算しておこう.1分の価値の考え方には色々あるが,
例えば,あなたのバイトの時給が900円だったとしたら,あなたにとっての1分は15円の価値を持つ,と考えることができる.
このとき,A軒の待ち時間30分は450円に,B亭の待ち時間10分は150円に相当する.
\item ここで求めた「待ち時間(を金銭換算したもの)」にラーメンの「料金」を加えたものが,
それぞれのラーメン屋を選ぶことの \textbf{\textbf{費用}} となる.時間価値が1分15円なら,それぞれのラーメン屋の費用は,
以下のように整理できる:
\begin{center}
\begin{tabular}{lll}
\hline
 & A軒 & B亭\\
\hline
待ち時間 & 450円(30分) & 150円(10分)\\
料金 & 500円 & 700円\\
\hline
費用 & 950円 & 850円\\
\hline
\end{tabular}
\end{center}
\item あなたにとっての各ラーメン屋の効用は,上記の価値と費用で表される.仮に,A軒とB亭の味・量に対する
あなたの価値がいずれも \(1,000\) 円だった場合,各選択肢の効用は以下のように整理できる:
\begin{center}
\begin{tabular}{lll}
\hline
 & A軒 & B亭\\
\hline
味・量に対する価値 & 1000円 & 1000円\\
待ち時間・料金に対する費用 & 950円 & 850円\\
\hline
効用 & 50円 & 150円\\
\hline
\end{tabular}
\end{center}

経済学の基本的考えでは「合理的な経済主体は,このように各選択肢の効用を評価した上で,
最も効用が高い選択肢を選択する」と仮定する.
\end{itemize}
\end{itemize}

\subsection{}
\label{sec:org774f0fd}
\subsection{}
\label{sec:org42c7f4a}
\begin{itemize}
\item 経済学というのは,つまるところ \textbf{\textbf{資源の配分}} の学問.
\item たとえば,レストラン\footnote{もちろん,居酒屋でも良いのだが,未成年の受講生に配慮して.}の経営を考えてみよう.そこでは,限られた数の厨房スタッフが,限られた調理器具や設備を使って,限られた食材・調味料から料理を作り,それを限られた数のホールスタッフがお客さんのテーブルに提供することで,対価としての報酬を得ている.さらに,その報酬(これもまた有限の資源である)を,スタッフの給料や,仕入れや,設備の維持更新などに充足している.これらをデタラメやったらすぐに店がつぶれてしまうから,経営者は何とかやりくりしないといけない.
\item あなたがバイト先としてレストランを考えているとしよう.
\item あなたが今夜の夕食にどのレストランを選ぶかを考えてみよう.あなたは,お財布の中身(限られた予算)と相談しながら,美味しくて,雰囲気が良くて,スタッフの器量がよいレストランを選ぶだろう.もちろん,夕食に充てられる時間も限られているから,注文した料理がいつまでも出てこないレストランは選ばない.
\item 経営者の経済活動:入力(生産要素)として,労働力(厨房スタッフ,ホールスタッフ)や食材を投入し,生産設備(厨房,調理器具)を使って生産した財(料理)を提供する.
\item 
\end{itemize}
\section{線形計画問題:リア充問題}
\label{sec:orgfc145f6}
\begin{itemize}
\item 時間制約,予算制約,体力制約の下で,充実度(効用)を最大化するようにバイトとデートの時間の配分を決める.
\item 定式化:未知変数,制約条件,目的関数.
\item 線形計画問題:目的関数が未知変数に対して線形で,制約条件が未知変数に対する線形不等式.
\item 幾何学的なイメージとして理解してみる.
\item 幾何学的なイメージのまま,感度分析してみる:バイト給料が下った場合,彼女と喧嘩した場合,試験前でリア充時間が減った場合.
\item 標準最小化問題.どんな問題でもこの形に直せば,システマティックに分析できるよ.
\end{itemize}
\section{}
\label{sec:orgaf585c7}

\section{}
\label{sec:orgeeef415}
\end{document}